\documentclass{article}

\usepackage[utf8]{inputenc}
\usepackage{german}


\begin{document}
	
	\tableofcontents
	
	\newpage
	
	\section{Einführung}
	
	\subsection{Künstliche Intelligenz}
	
	\subsection{Evolutionäre Algorithmen}
	
	
	\subsubsection{Neuronale Netze}
	
	
	\subsection{Der Ansatz der mutagenen Netzwerke}
	
	Bei konventionellen evolutionären Algorithmen werden die Mutationsvektoren der neuronalen Netzwerke meist über einen Gaußschen Zufallsgenerator generiert. Dieses Vorgehen ist, wenn die einzelnen Faktoren w\textsubscript{n} unabhängig voneinander auf die Fitnessfunktion $\varphi(\vec{w})$ einwirken. Bei neuronalen Netzen allerdings hängt das Funktionieren eines Merkmals meist von sehr vielen Elementen ab (vgl. Polygenie in der Biologie).
	\\\\
	Wenn man möglichst viele neue Eigenschaften generieren will, liegt es also nahe, Algorithmen zu schreiben, die solche Merkmalsmuster zufällig generieren.
	\\\\
	Allerdings liegt hier natürlich die Idee nahe, weitere neuronale Netze, die auch evolutionäre Prozesse durchlaufen zur Mutationsgeneration hinzuzuziehen.
	
	\newpage
	\section{Detailliertes Konzept}
	
	Wenn ein neuronales Netzwerk $\vec{w}$ mit dem Mutationsvektor $\Delta\vec{w}$, der von einem weiteren Netzwerk generiert werden soll, muss der Outputvektor $\vec{a_m}$  des Mutationsgenerationsnetzwerkes (von jetzt an: "`mutagenes Netzwerk"') $\vec{w_m}$ so groß wie das gesamte Netzwerk $\vec{w}$ sein.
	\\\\
	Jedem neuronalen Netz wird nun ein mutagenes Netzwerk zugewiesen. Die mutagenen Netzwerke unterlaufen nun jede vierte Generation den Selektions-Mutations-Reproduktionsprozess, wobei die Fitnessfunktion der mutagenen Netzwerke der Fitness der von den mutagenen Netzwerken mutierten normalen Netzwerke entspricht.
	
	
	
	
	
\end{document}